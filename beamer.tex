\documentclass{beamer}

\usepackage[utf8]{inputenc}
\usepackage{default}
 \usepackage{graphicx}

\newtheorem{definicion}{Definición}
\newtheorem{ejemplo}{Ejemplo}

\begin{document}
%%%%%%%%%%%%%%%%%%%%%%%%%%%%%%%%%%%%%%%%%%%%%%%%%%%%%%%%%%%%%%%%%%%%%%%%%%%%%%%
\title[Presentación con Beamer]{Numero Pi }
\author[Alba De León Hdez]{Alba De León Hdez}
\date[23-04-2014]{23 de abril de 2014}
%%%%%%%%%%%%%%%%%%%%%%%%%%%%%%%%%%%%%%%%%%%%%%%%%%%%%%%%%%%%%%%%%%%%%%%%%%%%%%%





  
%++++++++++++++++++++++++++++++++++++++++++++++++++++++++++++++++++++++++++++++
\begin{frame}
  \titlepage
  \begin{small}
    \begin{center}
     Facultad de Matemáticas \\
     Universidad de La Laguna
    \end{center}
  \end{small}

\end{frame}
%++++++++++++++++++++++++++++++++++++++++++++++++++++++++++++++++++++++++++++++

%++++++++++++++++++++++++++++++++++++++++++++++++++++++++++++++++++++++++++++++
\begin{frame}
  \frametitle{Índice}
  \tableofcontents[pausesections]
\end{frame}
%++++++++++++++++++++++++++++++++++++++++++++++++++++++++++++++++++++++++++++++


\section{Primera Sección}


%++++++++++++++++++++++++++++++++++++++++++++++++++++++++++++++++++++++++++++++
\begin{frame}

\frametitle{Número Pi}

\begin{definicion}
$\pi$ (pi) es la relación entre la longitud de una circunferencia y su diámetro, en geometría euclidiana. 
Es un número irracional y una de las constantes matemáticas más importantes. 
Se emplea frecuentemente en matemáticas, física e ingeniería. 

\end{definicion}

\end{frame}
%++++++++++++++++++++++++++++++++++++++++++++++++++++++++++++++++++++++++++++++

\section{Segunda Sección}

%++++++++++++++++++++++++++++++++++++++++++++++++++++++++++++++++++++++++++++++
\begin{frame}

\frametitle{Valor del numero Pi}
\begin{definicion}
El valor de $\pi$se ha obtenido con diversas aproximaciones a lo largo de la historia, siendo una 
de las constantes matemáticas que más aparece en las ecuaciones de la física, junto con el número e. 
Cabe destacar que el cociente entre la longitud de cualquier circunferencia y la de su diámetro 
no es constante en geometrías no euclídeas.
\end{definicion}

\end{frame}
%++++++++++++++++++++++++++++++++++++++++++++++++++++++++++++++++++++++++++++++
\section{Tercera Sección}
  \begin{frame}
    \frametitle{Formula}
      \begin{itemize}
	\item
	  Primera formula:
	  \[ 
	  x=\frac{a_2 x^2 + a_1 x + a_0}{1+2z^3}, 
	  \quad x+y^{2n+2}=\sqrt{b^2-4ac}
	  \]
	\pause

	\item
	  Segunda formula:
	  \[ S_n=a_1+\cdots + a_n = \sum_{i=1}^n a_i \]
	\pause

	\item
	  Tercera formula:
	  \[
	  \int_{x=0}^{\infty} x\,\text{e}^{-x^2}
	  \text{d}x=\frac{1}{2},\quad\text{e}^{i\pi}+1=0
	  \]
	\pause
	
	\item
	  Cuarta formula:
	  
	   r: A = $\pi$ * r^2
	\pause
  
	\item
	  Quinta formula:
	  
	  r: C = 2 * $\pi$ * r
	  
	\pause
      \end{itemize}
  \end{frame}

%++++++++++++++++++++++++++++++++++++++++++++++++++++++++++++++++++++++++++++++

\section{Bibliografía}
%++++++++++++++++++++++++++++++++++++++++++++++++++++++++++++++++++++++++++++++
\begin{frame}
  \frametitle{Bibliografía}

  \begin{thebibliography}{10}
    \beamertemplatebookbibitems
    \bibitem[Número Pi]{Pi}
  
    \beamertemplatebookbibitems
    \bibitem[Número Pi2]{Pi2}
   
    {\small http://es.wikipedia.org/}

    {\small http://es.wikipedia.org/wiki/Pi}

  \end{thebibliography}
\end{frame}

%++++++++++++++++++++++++++++++++++++++++++++++++++++++++++++++++++++++++++++++

\end{document}